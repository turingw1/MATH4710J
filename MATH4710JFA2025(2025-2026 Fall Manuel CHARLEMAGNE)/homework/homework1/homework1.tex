% MATH471 Homework LaTeX Template
% Compile with: pdflatex MATH471_homework_template.tex (or xelatex/lualatex)
\documentclass[11pt,a4paper]{article}

% Packages
\usepackage[utf8]{inputenc}
\usepackage[T1]{fontenc}
\usepackage{geometry}
\geometry{left=25mm,right=25mm,top=25mm,bottom=25mm}
\usepackage{amsmath,amssymb,amsthm}
\usepackage{mathtools}
\usepackage{enumitem}
\usepackage{graphicx}
\usepackage{float}
\usepackage{xcolor}
\usepackage{fancyhdr}
\usepackage{tikz}
\usepackage{caption}
\usepackage{booktabs}
\usepackage{listings}
\usepackage[linesnumbered,ruled,longend]{algorithm2e}
\usepackage[colorlinks=true,linkcolor=blue]{hyperref}
\SetKwInOut{Input}{Input}
\SetKwInOut{Output}{Output}
\SetKwProg{Fn}{Function}{\string:}{end}
\SetKwFunction{algohw}{AlgoHw}
% Theorem environments
\newtheorem{theorem}{Theorem}[section]
\newtheorem{lemma}[theorem]{Lemma}
\newtheorem{corollary}[theorem]{Corollary}
\theoremstyle{definition}
\newtheorem{definition}[theorem]{Definition}
\newtheorem{remark}[theorem]{Remark}

% Homework-specific environments
\newcounter{problem}
\newenvironment{problem}[1][]{\refstepcounter{problem}\par\medskip
\noindent{\bf Problem~\theproblem.}\ifx\relax#1\relax\else\ (#1)\fi\newline
}{\medskip}

\newenvironment{solution}{\par\noindent{\it Solution.} }{\hfill $\square$ \par}

% Header / Footer
\pagestyle{fancy}
\fancyhf{}
\lhead{MATH4710J --- Numerical Methods}
\chead{}
\rhead{Homework \#\hwnumber}
\lfoot{Student: \studentname}
\cfoot{\thepage}
\rfoot{October 1st, 2025}

% User commands (edit these in document body)
\newcommand{\coursenum}{MATH4710J}
\newcommand{\coursename}{Numerical Methods}
\newcommand{\instructor}{Manuel CHARLEMAGNE}
\newcommand{\studentname}{Turingw1}
\newcommand{\studentid}{000000}
\newcommand{\hwnumber}{1}

% Shortcuts
\newcommand{\R}{\mathbb{R}}
\newcommand{\N}{\mathbb{N}}
\newcommand{\abs}[1]{\left\lvert#1\right\rvert}
\newcommand{\norm}[1]{\left\lVert#1\right\rVert}

% Document
\begin{document}

% Title block (edit fields below)
\begin{center}
  {\Large \textbf{Homework \#\hwnumber}}\\[6pt]
  {\large \textbf{\coursenum\ --- \coursename}}\\[4pt]
  {Instructor: Manuel CHARLEMAGNE}\\[4pt]
  {Student: \studentname\ (ID:\@ \studentid)}\\[4pt]
  {Due date: October 1st, 2025}
\end{center}

\vspace{8pt}
\hrule
\vspace{12pt}

% Example problems
\begin{problem}
\begin{itemize}
\item[1. ] Prove that $\mathbb{N}$, $\mathbb{Z}$, and $\mathbb{Q}$ have the same number of elements.
\item[2. ] Prove that [0, 1] has as many elements as $\mathbb{R}$.
\item[3. ] Prove that [0, 1] has more elements than $\mathbb{N}$. Hint: understand Cantor's diagonal argument.
\end{itemize}
\begin{solution}
\begin{itemize}
  \item[1. ] \textbf{$\mathbb{N}$, $\mathbb{Z}$, and $\mathbb{Q}$ have the same number of elements}\\
 We want to show that the sets of natural numbers $\mathbb{N}$, integers $\mathbb{Z}$, and rational numbers $\mathbb{Q}$ are all countably infinite, i.e., there exists a bijection between each pair.

\textbf{Step 1: $\mathbb{N}$ and $\mathbb{Z}$ have the same cardinality.}

Define a function $f: \mathbb{N} \to \mathbb{Z}$ by
\[
f(n) = \begin{cases}
\phantom{-}\frac{n}{2} & \text{if $n$ is even} \\
-\frac{n-1}{2} & \text{if $n$ is odd}
\end{cases}
\]
This function lists the integers in the order $0, 1, -1, 2, -2, 3, -3, \ldots$ as $n$ increases. It is a bijection, so $|\mathbb{N}| = |\mathbb{Z}|$.
\textbf{Step 2: $\mathbb{N}$ and $\mathbb{Q}$ have the same cardinality.}

Every rational number can be written as $p/q$ with $p \in \mathbb{Z}$, $q \in \mathbb{N}$, and $\gcd(p,q)=1$. Arrange all such pairs $(p,q)$ in a grid and enumerate them diagonally (Cantor's diagonal argument), skipping duplicates and zero denominators. This process lists all rationals in a sequence, showing that $\mathbb{Q}$ is countable.

Therefore, there exist bijections $\mathbb{N} \leftrightarrow \mathbb{Z} \leftrightarrow \mathbb{Q}$, so all three sets have the same number of elements.
\item[2.] \textbf{[0, 1] has as many elements as $\mathbb{R}$}\\
To show that the interval [0, 1] has the same cardinality as $\mathbb{R}$, we can construct a bijection between these two sets.
Consider the function $f: (0, 1) \to \mathbb{R}$ defined by
\[f(x) = \tan\left(\pi x - \frac{\pi}{2}\right)\]
This function maps the open interval (0, 1) onto the entire real line $\mathbb{R}$. To include the endpoints 0 and 1, we can extend this function by defining:
\[f(0) = -\infty, \quad f(1) = +\infty\]
This shows that there is a bijection between [0, 1] and $\mathbb{R}$, hence they have the same cardinality.

\item[3.] \textbf{[0, 1] has more elements than $\mathbb{N}$}\\
To show that the interval [0, 1] has more elements than $\mathbb{N}$, we can use Cantor's diagonal argument.
Assume, for the sake of contradiction, that there is a bijection between $\mathbb{N}$ and [0, 1]. This means we can list all numbers in [0, 1] as a sequence:
\[
x_1, x_2, x_3, \ldots
\]
where each $x_i$ is represented in its decimal expansion. Now, we construct a new number $y$ in [0, 1] by changing the $i$-th digit of $x_i$ to a different digit (for example, if the $i$-th digit is 5, change it to 6; if it's 9, change it to 0, etc.). This new number $y$ differs from each $x_i$ in at least one decimal place, meaning $y$ cannot be in the list.
This contradiction shows that there is no bijection between $\mathbb{N}$ and [0, 1], hence [0, 1] has more elements than $\mathbb{N}$.
\end{itemize}
\end{solution}
\end{problem}

\begin{problem}
\begin{itemize}
  \item[1.] Prove the Cauchy-Schwarz inequality (1.20|1.39) over the complex numbers.
  \item[2.] Show that a distance is always positive.
\end{itemize}
\begin{solution}
  \begin{itemize}
    \item[1.] \textbf{Cauchy-Schwarz Inequality:}
\[
\bigl|\langle x,y\rangle\bigr| \le \|x\|\,\|y\|,\qquad x,y\in V,
\]

\textbf{Proof:}
\begin{align}
0 &\le \langle x-\lambda y,\;x-\lambda y\rangle \\
  &= \langle x,x\rangle -\overline{\lambda}\langle y,x\rangle -\lambda\langle x,y\rangle + |\lambda|^2\langle y,y\rangle.
\end{align}
Take $\displaystyle \lambda=\frac{\langle y,x\rangle}{\langle y,y\rangle}$ (when $y\neq 0$), we have
\[
0 \le \langle x,x\rangle - \frac{|\langle x,y\rangle|^2}{\langle y,y\rangle},
\]
from which it follows that  
\[
|\langle x,y\rangle|^2 \le \langle x,x\rangle\,\langle y,y\rangle,
\]
and hence
\[
|\langle x,y\rangle| \le \|x\|\,\|y\|.
\]

which is the Cauchy-Schwarz inequality.
\item[2. ]
\textbf{Distance is Always Positive:}

A distance function $d: X \times X \to \mathbb{R}$ on a set $X$ must satisfy the following properties for all $x, y, z \in X$:
\\1. $d(x, y) = 0$ if and only if $x = y$.
\\2. Symmetry: $d(x, y) = d(y, x)$.
\\3. Triangle inequality: $d(x, z) \leq d(x, y) + d(y, z)$.
\\
\textbf{Proof}
\\
\[\forall x, y \in X, d(x, y) + d(y, x) = 2 d(x, y) \geq d(x, x) = 0\]
\end{itemize}
\end{solution}
\end{problem}

% If you have multiple parts
\begin{problem}
\begin{enumerate}
  \item  Let $f$ be a linear map from a vector space $V_1$ into a vector space $V_2$. Show that the dimension
of $V_1$ is the sum of the dimensions of the kernel and of the image of $f$. This result is called the
\textbf{rank-nullity theorem}.
  \item  Prove that the composition of two linear maps is a linear map.
  \item  Prove that the inverse of a linear map is a linear map.
\end{enumerate}
\begin{solution}
\begin{enumerate}
  \item Let $f: V_1 \to V_2$ be a linear map. We want to show that
  \[\dim(V_1) = \dim(\ker(f)) + \dim(\text{im}(f)).\]
  Let $\{v_1, v_2, \ldots, v_k\}$ be a basis for $\ker(f)$. We can extend this basis to a basis for $V_1$ by adding vectors $\{v_{k+1}, v_{k+2}, \ldots, v_n\}$ such that $\{v_1, v_2, \ldots, v_n\}$ is a basis for $V_1$. The images of the added vectors under $f$, $\{f(v_{k+1}), f(v_{k+2}), \ldots, f(v_n)\}$, form a basis for $\text{im}(f)$ (Since $\{f(v_{k+1}), f(v_{k+2}), \ldots, f(v_n)\}$ can't be 0 unless $v_{k+1}, v_{k+2}, \ldots, v_n$ are in $\ker(f)$). Therefore, we have
  \[\dim(V_1) = n = k + (n - k) = \dim(\ker(f)) + \dim(\text{im}(f)).\]
  \item Let $f: V_1 \to V_2$ and $g: V_2 \to V_3$ be linear maps. We want to show that the composition $g \circ f: V_1 \to V_3$ is linear. For any $u, v \in V_1$ and scalar $c$, we have
  \[(g \circ f)(u + v) = g(f(u + v)) = g(f(u) + f(v)) = g(f(u)) + g(f(v)) = (g \circ f)(u) + (g \circ f)(v),\]
  and
  \[(g \circ f)(cu) = g(f(cu)) = g(cf(u)) = c g(f(u)) = c (g \circ f)(u).\]
  Thus, $g \circ f$ is linear.
  \item Let $f: V_1 \to V_2$ be a bijective linear map. We want to show that the inverse map $f^{-1}: V_2 \to V_1$ is linear. For any $f(u), f(v) \in V_2$ and scalar $c$, we have
  \[f^{-1}(f(u) + f(v)) = f^{-1}(f(u + v)) = u + v = f^{-1}(f(u)) + f^{-1}(f(v)) ,\]
  and
  \[f^{-1}(c(f(u))) = f^{-1}(f(cu)) = cu = c f^{-1}(f(u)).\]
  And since $f$ is bijective, every element in $V_2$ can be written as $f(u)$ for some $u \in V_1$.
  Thus, $f^{-1}$ is linear.
  
\end{enumerate}
\end{solution}
\end{problem}


\begin{problem}
Intuitively a complete space has “no point missing” anywhere. In particular it means that any Cauchy
sequence converges inside the space. In this exercise we show that $e$ is not rational while we can find a
Cauchy sequence of rationals converging to $e$.
\begin{enumerate}
  \item Show that $e$ is irrational.
  \item Show that the sequence ${(u_n)}_{n \in \mathbb{N}}$ defined by $u_n = {(1 + \frac{1}{n})}^n$ is a Cauchy sequence converging to $e$.
  \item Is $\mathbb{Q}$ complete? Explain.
\end{enumerate}
\begin{solution}
\begin{enumerate}
  \item To show that $e$ is irrational, we can use a proof by contradiction. Assume that $e$ is rational, i.e., $e = \frac{p}{q}$ for some integers $p$ and $q$. Consider the series expansion of $e$:
  \[
  e = \sum_{n=0}^{\infty} \frac{1}{n!} = 1 + 1 + \frac{1}{2!} + \frac{1}{3!} + \cdots
  \]
  Multiplying both sides by $q$!, we get
  \[
  q! e = q! \left(1 + 1 + \frac{1}{2!} + \frac{1}{3!} + \cdots\right) = q! + q! + \frac{q!}{2!} + \frac{q!}{3!} + \cdots
  \]
  The left side is an integer since $e = \frac{p}{q}$ and $q! e = p (q-1)$!. However, the right side is not an integer because the terms $\frac{q!}{n!}$ for $n > q$ are not integers. This leads to a contradiction, hence $e$ is irrational.
  
  \item
Let
\[
a_n=\Bigl(1+\frac{1}{n}\Bigr)^n,\qquad n\ge1,
\]
and define the function
\[
f(x)=x\ln\!\Bigl(1+\frac{1}{x}\Bigr),\qquad x\ge1.
\]
Notice that for integer $n$ we have $f(n)=\ln a_n$.
To show that $\{a_n\}$ is increasing, it suffices to prove that $f$ is increasing on $[1,\infty)$.

Let $u=1/x>0$. Then
\[
f'(x)=\ln(1+u)+x\cdot\frac{1}{1+u}\cdot\frac{du}{dx}.
\]
Since $u=1/x$, we have $du/dx=-1/x^2=-u^2$. Substitution yields
\[
f'(x)=\ln(1+u)-\frac{u}{1+u}.
\]
Define $g(u)=\ln(1+u)-\tfrac{u}{1+u}$. Then
\[
g'(u)=\frac{1}{1+u}-\frac{1}{(1+u)^2}=\frac{u}{(1+u)^2}>0 \qquad (u>0),
\]
and $g(0)=0$. Hence $g(u)>0$ for all $u>0$, so $f'(x)>0$ for $x>0$. Therefore $f$ is increasing, and thus $\ln a_n=f(n)$ is increasing in $n$, i.e. $\{a_n\}$ is monotone increasing.

Next, we provide an upper bound. Using the inequality $\ln(1+t)\le t$ valid for $t>-1$, we obtain for $t=1/n$:
\[
\ln a_n = n\ln\Bigl(1+\frac{1}{n}\Bigr) \le n\cdot\frac{1}{n}=1.
\]
Therefore $a_n\le e^1=e$ for all $n$. Consequently, the sequence $\{a_n\}$ is monotone increasing and bounded above by $e$. By the monotone convergence theorem for real numbers, $\{a_n\}$ converges in $\mathbb{R}$.

Finally, in the metric space $(\mathbb{R},|\cdot|)$, every convergent sequence is Cauchy. Hence $\{a_n\}$ is a Cauchy sequence.

\item The set of rational numbers $\mathbb{Q}$ is not complete. A metric space is said to be complete if every Cauchy sequence in that space converges to a limit that is also within the same space. However, there exist Cauchy sequences of rational numbers that converge to irrational numbers, which are not in $\mathbb{Q}$. For example, the sequence defined by $a_n = (1 + \frac{1}{n})^n$ converges to $e$, which is irrational. Since $e \notin \mathbb{Q}$, this shows that $\mathbb{Q}$ is not complete. 
\\
For simpler example, consider the sequence defined by $b_n = 3.1, 3.14, 3.141, 3.1415, \ldots$, which converges to $\pi$. Since $\pi$ is irrational and not in $\mathbb{Q}$, this is another example of a Cauchy sequence in $\mathbb{Q}$ that does not converge to a limit in $\mathbb{Q}$. Thus, $\mathbb{Q}$ is not a complete metric space.

\end{enumerate}
\end{solution}
\end{problem}

\begin{problem}
  1. Write the pseudocode for at least one the following strategy to approximate $\pi$.
  \begin{enumerate}[label = (\alph*)]
    \item The polygons method;
    \item Machin's formula $\frac{\pi}{4} = 4 \arctan \frac{1}{5} - \arctan \frac{1}{239}$
    and Taylor series;
  \end{enumerate}
2. Implement at least one of the previous algorithms in MATLAB.
\end{problem}\\

\begin{solution}\\

\begin{enumerate}
  \item 
  \begin{itemize}
  \item[a)] For the unit circle (radius 1), the side length of an inscribed regular $n$-gon is
\[
c_n = 2\sin\frac{\pi}{n},\quad\text{so}\quad p_n = n c_n = 2n\sin\frac{\pi}{n}
\]
is the inscribed polygon perimeter. The perimeter of the circumscribed regular $n$-gon is
\[
P_n = 2n\tan\frac{\pi}{n}.
\]
Thus
\[
n\sin\frac{\pi}{n}\le \pi \le n\tan\frac{\pi}{n}.
\]

% Doubling formula for the inscribed polygon side
If we double the number of sides, the inscribed side lengths satisfy
\[
c_{2n}=2\sin\frac{\pi}{2n}= \sqrt{\,2-2\sqrt{\,1-\dfrac{c_n^2}{4}\,}\, }.
\]
Consequently
\[
p_{2n}=2n\,c_{2n}.
\]
Repeating the doubling step yields arbitrarily tight bounds for $\pi$.\\

\begin{algorithm}[H]
\Input{number of sides $n$ (initially 6), number of iterations $k$}
\Output{approximation of $\pi$}
\BlankLine
\Fn{\algohw{n, k}}{
$c_n \leftarrow 2\sin(\pi/n)$\;
$p_n \leftarrow n \cdot c_n$\;
\For{$i \leftarrow 1$ \KwTo $k$}{
$c_n \leftarrow \sqrt{2 - 2\sqrt{1 - c_n^2/4}}$\;
$p_n \leftarrow 2n \cdot c_n$\;
$n \leftarrow 2n$\;
}
\KwRet{$p_n $}
}
\caption{Polygon method to approximate $\pi$}
\end{algorithm}

  \item[b)] Using Machin's formula and Taylor series to approximate $\pi$:
\[\frac{\pi}{4} = 4 \arctan \frac{1}{5} - \arctan \frac{1}{239}\]
The Taylor series expansion for $\arctan(x)$ is given by:
\[\arctan(x) = \sum_{n=0}^{\infty} \frac{(-1)^n x^{2n+1}}{2n+1}\]
To approximate $\pi$, we can truncate the series after a finite number of terms. Here is the pseudocode for the algorithm:\\
\begin{algorithm}[H]
\Input{number of terms $N$}
\Output{approximation of $\pi$}
\BlankLine
\Fn{\algohw{N}}{
$\arctan(\frac{1}{5}) \leftarrow 0$\;
$\arctan(\frac{1}{239}) \leftarrow 0$\;
\For{$n \leftarrow 0$ \KwTo $N-1$}{
$\arctan(\frac{1}{5}) \leftarrow \arctan(\frac{1}{5}) + \frac{(-1)^n (1/5)^{2n+1}}{2n+1}$\;
$\arctan(\frac{1}{239}) \leftarrow \arctan(\frac{1}{239}) + \frac{(-1)^n (1/239)^{2n+1}}{2n+1}$\;
}
$\pi \leftarrow 4 \cdot (4 \cdot \arctan(\frac{1}{5}) - \arctan(\frac{1}{239}))$\;
\KwRet{$\pi$}
}
\caption{Machin's formula to approximate $\pi$}
\end{algorithm}
\end{itemize}
  \item Here is a MATLAB implementation of the polygon method to approximate $\pi$:
\begin{lstlisting}[language=Matlab, caption=Polygon Method to Approximate Pi]
function pi_approx = approximate_pi_polygon(n, k)
    % n: initial number of sides (e.g., 6)
    % k: number of iterations (doublings)
    c_n = 2 * sin(pi / n); % side length of inscribed polygon
    p_n = n * c_n;         % perimeter of inscribed polygon
    for i = 1:k
        c_n = sqrt(2 - 2 * sqrt(1 - c_n^2 / 4));
        p_n = 2 * n * c_n;
        n = 2 * n;
    end
    pi_approx = p_n;
end
\end{lstlisting}
\end{enumerate}
\end{solution}
\vfill
\noindent{\small Compiled with \LaTeX.}
\end{document}
